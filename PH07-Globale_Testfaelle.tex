\section{Globale Testf�lle}

\comment{Was sind typische Szenarien, die das Produkt erf�llen muss?}

Jede Produktfunktion \textit{/F????/} wird anhand von konkreten Testf�llen \textit{/T????/} getestet.\\
Die dabei verwendeten Namen werden rein zuf�llig gew�hlt.

\begin{description}
  \item[/T0010/]
    \textit{Registrieren:}
    Herr Tim Testmann registriert sich mit dem gew�nschten Benutzernamen \textit{testmann}
    und dem Passwort \textit{testtest} und der EMailadresse \textit{tim@testmann.de} am System.\\
    Frau Beate Betamuster registriert sich ebenfall mit dem gew�nschten Benutzernamen \textit{betamuster}
    und dem Passwort \textit{betabeta} und der EMailadresse \textit{info@stefan-baur.de} am System.
  \item[/T0020/]
    \textit{Anmelden:}
    Tim Testmann meldet sich am System unter Benutzung seines Benutzernamens und Passwortes am System an.
  \item[/T0030/]
    \textit{Abmelden:}
    Tim Testmann meldet sich vom System wieder ab.
  \item[/T0040/]
    \textit{Kennung anfordern:}
    Beate Betamuster hat ihre Kennung vergessen und fordert unter Angabe ihrer EMailadresse ihre Kennung an.
  \item[/T0050/]
    \textit{Passwort �ndern:}
    Beate Betamuster �ndert sodann ihr Passwort \textit{betabeta} in \textit{testbeta} ab.
  \item[/T????/]
    ...
\end{description}

\textcolor{warning}{Es muss zu jeder weiteren Produktfunktion ein konkreter Testfall hinzugef�gt werden ...}
