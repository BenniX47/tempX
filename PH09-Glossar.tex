\section{Glossar}

\comment{Definition aller wichtigen Begriffe zur Sicherstellung einer einheitlichen Terminologie.}

\begin{description}
  \item[Fernspiele]
    sind Spiele, die eine Bedenkzeit von mindestens einen Tag haben. Beim Fernschach oder Briefschach haben die Spieler f�r jeden
    Zug viele Tage Bedenkzeit.
  \item[Hauptseite]
    ist die Seite, auf die der Benutzer kommt, wenn er sich erfolgreich am System angemeldet hat.
  \item[Kennung]
    ist das Tupel, das ein Benutzer zur Anmeldung an das System ben�tigt: \textit{Benutzername} und \textit{Passwort}.
  \item[Konfiguration]
    Jeder einzelne Benutzer kann seine Nutzungsoberfl�che individuell gestalten.
    Diese Einstellung wird als \textit{pers�nliche Konfiguration} bezeichnet.\\
    Der Administrator konfiguriert das System f�r alle Benutzer \textit{(Konfiguration des Administrators)}.
  \item[�ffentlich]
    Unter \textit{�ffentlich} versteht man die Lesbarkeit nur innerhalb der Spielgemeinschaft des Systems,
    soweit nicht n�her beschrieben.
  \item[Portfolio]
    ist die Sammelmappe bzw. die Dokumentenmappe eines Benutzers.
    Darin kann der Benutzer all seine f�r wichtig empfundenen Informationen aufnehmen,
    wie z.B. interessante Spiele oder starke Gegner, sowie Notizen und Nachrichten.
    Das Portfolio kann als pers�nliche Datenbank betrachtet werden.
  \item[registrieren]
    Erstmaliges Anmelden eines beliebigen Internet-Benutzers, f�r den noch keine Kennung f�r das System vorliegt.
  \item[Spielgemeinschaft]
    Die Menge aller spielenden Benutzer des Systems.
  \item[Startseite]
    bzw. Loginseite ist die Seite die angezeigt wird, wenn ein beliebiger Internet-Benutzer auf den Internetdienst gelangt.
  \item[System]
    ist ein Synonym f�r den Internetdienst \textit{Brettspiele}, soweit nicht n�her beschrieben.
  \item[Verkehrssprache]
    ist die Sprache des Systems wie z.B. zur Kommunikation zwischen Benutzern.
\end{description}
