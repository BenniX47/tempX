\documentclass[a4paper, 11pt, ngerman, fleqn]{article}
\usepackage[ansinew]{inputenc}
\usepackage{babel}
\usepackage{ngerman}
\usepackage{coordsys,logsys,color}
\usepackage{german,fancyhdr}
\usepackage{hyperref}
\usepackage{texdraw}
\input{txdtools}

\NeedsTeXFormat{LaTeX2e}
\ProvidesPackage{hyperref}
\definecolor{darkblue}{rgb}{0,0,.6}
\hypersetup{pdftex=false, colorlinks=true, breaklinks=true, linkcolor=darkblue, menucolor=darkblue, pagecolor=darkblue, urlcolor=darkblue, citecolor=darkblue}

\pagestyle{fancy}

\input{_Pflichtenheft.tex}

\begin{document}
  \lhead{\sc{Beispiel: Pflichtenheft Brettspiele}}
%\cfoot{-~\thepage~-}
\title{Beispiel: Pflichtenheft}
\project{Brettspiele}
\version{1.2}
\prerequisite{Lastenheft}
\prerequisiteref{http://www.stefan-baur.de/downloads/Lastenheft.pdf}
\author{Stefan K. Baur}
\home{www.stefan-baur.de}
\homeref{http://www.stefan-baur.de/cs.se.pflichtenheft.html}

\maketitle

\textcolor{warning}{Diese Datei zeigt NUR ein Beispiel eines Pflichtenheftes!}\\
\textcolor{warning}{Verwendung auf eigene Gefahr!}
%\thispagestyle{empty}
%~
%\newpage
%\tableofcontents
 \newpage
  \tableofcontents \newpage
  \section{Zielbestimmungen}

\comment{Welche Musskriterien, Wunschkriterien, Abgrenzungskriterien sind erforderlich?}

\textbf{Brettspiele} stellt einen Internetdienst dar, der das Spielen von Brettspielen zwischen Einzelpersonen erm�glicht.
Im Folgenden bezeichne \textit{Benutzer} und \textit{Spieler} sowohl die weiblichen, als auch die m�nnlichen Benutzer und Spieler.

\subsection{Musskriterien}

\begin{itemize}
  \item Der Benutzer-Account
    \begin{itemize}
      \item Der Internet-Benutzer kann sich selbst am System registrieren.
      \item Der Benutzer kann sich am System anmelden und vom System abmelden.
      \item Der Benutzer kann seine Kennung anfordern.
      \item Der Benutzer kann seine pers�nlichen Daten einsehen und �ndern, sowie deren Sichtbarkeit innerhalb der Spielgemeinschaft einstellen.
      \item Der Benutzer kann die pers�nlichen Daten anderer Benutzer einsehen, soweit diese sichtbar sind.
      \item Der Benutzer kann sein pers�nliches Profil einsehen, sowie die Sichtbarkeit innerhalb der Spielgemeinschaft einstellen.
      \item Der Benutzer kann das pers�nliche Profil anderer Benutzer einsehen, soweit dieses sichtbar ist.
      \item Der Benutzer kann seine Nutzungsoberfl�che nach eigenem Bedarf und Geschmack konfigurieren.
      \item Der Benutzer kann Konfigurationen der eigenen Nutzungsoberfl�che neu erstellen, speichern, l�schen, �ndern und wieder verwenden.
      \item Der Benutzer ist in Besitz einer eigenen Portfolio, in der er Konfigurationen, Benutzer, Spiele, Nachrichten und Notizen verwalten kann.
      \item Der Benutzer kann mit den Funktionen des Portfolios das System durchsuchen, die Suchergebnisse k�nnen dem Portfolio hinzugef�gt werden.
      \item Der Benutzer kann jeden einzelnen Eintrag im Portfolio kommentieren und bewerten.
      \item Die Benutzer k�nnen untereinander Nachrichten austauschen (Instant-Messaging).
    \end{itemize}
  \item Das Spiel
    \begin{itemize}
      \item Es stehen drei Spieltypen zur Verf�gung: \textit{M�hle}, \textit{Dame} und \textit{Schach}.
      \item Der Spieler kann Spiele beliebigen Typs er�ffnen.
      \item Der Spieler kann ein er�ffnetes Spiel aufnehmen.
      \item Die Spieler k�nnen sich untereinander zum Spiel herausfordern.
      \item Der Spieler muss gem�� den Spielregeln ziehen, ein Unentschieden anbieten/annehmen oder aufgeben bzw. gewinnen.
      \item Der Spieler kann bei jeder Spielaktion eine kurze Nachricht �bermitteln.
      \item Nach dem Beenden einer Partie wird das pers�nliche Profil gem�� der Spielst�rke des Spielers mit einer zum Spieltyp geh�renden Berechnungsfunktion aktualisiert.
    \end{itemize}
  \item Der Administrator
    \begin{itemize}
      \item Der Administrator konfiguriert die Betriebsparameter des Systems.
      %\item Der Administrator muss den Zugang aller Benutzer vor�bergehend sperren k�nnen,
      \item Der Administrator sichert die Datenbank.
    \end{itemize}
  \item Sonstiges
    \begin{itemize}
      \item Englisch als Verkehrssprache.
      \item Erweiterbarkeit des Systems weiterer europ�ischen Sprachen.
      \item Erweiterbarkeit des Systems weiterer Spieltypen \textit{(z.B. Go oder Backgammon)}.
      \item Erweiterbarkeit des Systems weiterer Bereiche \textit{(z.B. Spiele-Forum, Lehrb�cherverkauf)}.
      \item Erweiterbarkeit des Systems zur geb�hrenpflichten Nutzung f�r die Benutzer.
    \end{itemize}
\end{itemize}

\subsection{Wunschkriterien}

\begin{itemize}
  \item Die Benutzer k�nnen die Eintr�ge im Portfolio untereinander austauschen.
  \item Der Benutzer kann den Verlauf seiner beendeten Spiele einsehen.
  \item Zu jedem Spieltyp liegen die Spielregeln im System vor.
  \item Einfache Integration von Werbebannern, einstellbar durch Administrator.
\end{itemize}

\subsection{Abgrenzungskriterien}

\begin{itemize}
  \item Nur Fernspiele, also z.B. Fernschach und kein Blitzschach.
  \item Das System eignet sich nur f�r Zwei-Spieler-Spiele.
\end{itemize}
 \newpage
  \input{PH02-Produkteinsatz} \newpage
  \section{Produktumgebung}

\comment{Welche Software, Hardware und Orgware wird ben�tigt?}

Das Produkt ist weitgehend unabh�ngig vom Betriebssystem, sofern folgende Produktumgebung vorhanden ist.

\subsection{Software}

\begin{itemize}
  \item Client
    \begin{itemize}
      \item \textbf{www-Browser} der neuesten Generation: Internet Explorer 6 und Mozilla 1.3 \textit{(keine textbasierten Browser)}
    \end{itemize}
  \item Server
    \begin{itemize}
      \item \textbf{PHP} \textit{(mind. Version 4.0.5)}
      \item \textbf{MySQL}-Datenbank
      \item \textbf{SMTP}-Server \textit{(eMail-f�hig)}
    \end{itemize}
\end{itemize}

\subsection{Hardware}

\begin{itemize}
  \item Client
    \begin{itemize}
      \item Internetf�higer Rechner
    \end{itemize}
  \item Server
    \begin{itemize}
      \item Internetf�higer Server
      \item Rechner, der die Anspr�che der o.g. Server-Software erf�llt
      \item Ausreichend Rechen- und Festplattenkapazit�t
    \end{itemize}
\end{itemize}

\subsection{Orgware}

\begin{itemize}
  \item Gew�hrleistung der permanenten Internetanbindung
  \item Administrator muss den Internetdienst starten und die Betriebsparameter konfigurieren
\end{itemize}
 \newpage
  \input{PH04-Funktionale_Anforderungen} \newpage
  \section{Produktdaten}

\comment{Was speichert das Produkt (langfristig) aus Benutzersicht?}

Jeder Punkt \textbf{/D???/} stellt im Prinzip einen Datensatz dar.

\begin{description}
  \item[/D010/]
    \textit{Benutzerdaten:} Alle Informationen zu einem Benutzer:
    \begin{itemize}
      \item \textbf{BenutzerID} \textit{(eindeutig)}
      \item Kennung
        \begin{itemize}
          \item \textbf{Benutzername} \textit{(eindeutig)}
          \item \textbf{Passwort} \textit{(verschl�sselt)}
        \end{itemize}
      \item Pers�nliche Daten
        \begin{itemize}
          \item Informationen zur eigenen Person
            \begin{itemize}
              \item \textbf{Vorname}
              \item \textbf{Nachname}
              \item \textbf{Alter}
              \item \textbf{Geschlecht} \textit{(m�nnlich, weiblich)}
              \item kleines \textbf{Foto}
              \item \textbf{Begr��ungstext}
              \item \textbf{Slogan}
            \end{itemize}
          \item Kontaktinformationen
            \begin{itemize}
              \item \textbf{Stra�e und Hausnummer}
              \item \textbf{Postleitzahl}
              \item \textbf{Ort}
              \item \textbf{Land}
              \item \textbf{Telefon}
              \item \textbf{Fax}
              \item \textbf{eMail-Adresse} \textit{(eindeutig, g�ltig)}
              \item \textbf{Homepage}
            \end{itemize}
          \item \textbf{Sichtbarkeit} der einzelnen Eintr�ge der pers�nlichen Daten
        \end{itemize}
      \item Sonstige Daten
        \begin{itemize}
          \item \textbf{Registrierungsdatum} \textit{(Datum)}
          \item \textbf{letzte Anmeldung} \textit{(Datum)}
          \item \textbf{Besuchsz�hler}
          \item \textbf{Status} \textit{(Administrator, Benutzer)}
        \end{itemize}
    \end{itemize}
\end{description}

\begin{description}
  \item[/D020/]
    \textit{Profildaten:} Das pers�nliche Profil eines Benutzers zu einem Spieltyp.
    Die Anzahl der offenen, gewonnen, verlorenen und unentschiedenen Spiele
    wird zur Laufzeit ermittelt und nicht explizit im Profil gespeichert!
    \begin{itemize}
      \item \textbf{BenutzerID}
      \item \textbf{Spieletyp} \textit{(M�hle, Dame oder Schach)}
      \item \textbf{Wertungszahl}
      \item Sichtbarkeit
        \begin{itemize}
          \item \textbf{Offene Spiele}
          \item \textbf{Gewonnene Spiele}
          \item \textbf{Verlorene Spiele}
          \item \textbf{Unentschiedene Spiele}
        \end{itemize}
    \end{itemize}
\end{description}

\begin{description}
  \item[/D030/]
    \textit{Konfigurationsdaten:} Die pers�nliche Konfiguration eines Benutzers:
    \begin{itemize}
      \item \textbf{BenutzerID}
      \item Darstellungsfarben
        \begin{itemize}
          \item \textbf{Hintergrundfarbe}
          \item \textbf{Textfarbe}
          \item \textbf{Rahmenfarbe}
          \item \textbf{Spielfeldfarbe Weiss}
          \item \textbf{Spielfeldfarbe Schwarz}
        \end{itemize}
      \item Darstellungsschema
        \begin{itemize}
          \item \textbf{Menu} \textit{(links, oben oder rechts)}
          \item \textbf{Spielfeld} \textit{(zentrieren oder am Fensterrand)}
        \end{itemize}
      \item Ablauflogik
        \begin{itemize}
          \item vor Ausf�hrung eines Spielzuges Best�tigungsfenster anzeigen
          \item nach Ausf�hrung eines Spielzuges
            \begin{itemize}
              \item aktuelles Spiel erneut anzeigen oder
              \item alle eigenen offenen Spiele anzeigen oder
              \item das n�chste offene Spiel anzeigen
            \end{itemize}
        \end{itemize}
    \end{itemize}
\end{description}

\begin{description}
  \item[/D110/]
    \textit{Spieldaten:} Der aktuelle Status eines einzelnen Spieles:
    \begin{itemize}
      \item \textbf{SpielID} \textit{(eindeutig)}
      \item \textbf{Spieletyp} \textit{(M�hle, Dame oder Schach)}
      \item \textbf{Status} \textit{(Initiiert, Laufend, Weiss, Schwarz oder Unentschieden)}
      \item \textbf{Anzahl der Halbz�ge}
      \item Spieler 1 und 2 \textit{(Weiss und Schwarz)} jeweils:
        \begin{itemize}
          \item \textbf{BenutzerID}
          \item \textbf{Summe der Bedenkzeit} \textit{(X Minuten)}
        \end{itemize}
      \item Sonstige Daten
        \begin{itemize}
          \item \textbf{Halbzugbedenkzeit} \textit{(X Tage)}
          \item \textbf{Toleranzzeit} \textit{(X Tage)}
          \item \textbf{Initialisierungsdatum} \textit{(Datum)}
          \item \textbf{letzter Halbzug} \textit{(Datum)}
        \end{itemize}
    \end{itemize}
\end{description}

\begin{description}
  \item[/D111/]
    \textit{Zugdaten:} Einzelner Zug eines Spiels:
    \begin{itemize}
      \item \textbf{SpielID}
      \item \textbf{Zugnummer}
      \item \textbf{Anfangskoordinate}
      \item \textbf{Zielkoordinate}
    \end{itemize}
\end{description}
 \newpage
  \input{PH06-Nichtfunktionale_Anforderungen} \newpage
  \input{PH07-Globale_Testfaelle} \newpage
  \input{PH08-Systemmodelle} \newpage
  \input{PH08a-Szenarien} \newpage
  \input{PH08b-Anwendungsfaelle} \newpage
  \input{PH08c-Objektmodell} \newpage
  \input{PH08d-Dynamische_Modelle} \newpage
  \input{PH08e-Benutzerschnittstelle} \newpage
  \input{PH09-Glossar}
\end{document}
