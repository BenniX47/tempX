\documentclass[a4paper, 11pt, ngerman, fleqn]{article}
\usepackage[ansinew]{inputenc}
\usepackage{babel}
\usepackage{ngerman}
\usepackage{coordsys,logsys,color}
\usepackage{german,fancyhdr}
\usepackage{hyperref}
\usepackage{texdraw}
\input{txdtools}

\NeedsTeXFormat{LaTeX2e}
\ProvidesPackage{hyperref}
\definecolor{darkblue}{rgb}{0,0,.6}
\hypersetup{pdftex=false, colorlinks=true, breaklinks=true, linkcolor=darkblue, menucolor=darkblue, pagecolor=darkblue, urlcolor=darkblue, citecolor=darkblue}

\pagestyle{fancy}

\renewcommand{\familydefault}{cmss}

\definecolor{fgcgray}{rgb}{0.4, 0.4, 0.4}
\definecolor{warning}{rgb}{0.9, 0.1, 0.0}
\definecolor{bgctitle}{rgb}{0.5, 0.5, 0.5}
\definecolor{fgctitle}{rgb}{0.95, 0.95, 0.95}
\newcommand{\titlefont}[1]{\textcolor{fgctitle}{\fontfamily{cmss}\fontseries{bx}\fontshape{n}\fontsize{20.48}{0pt} \selectfont #1}}
\newcommand{\inversetitlefont}[1]{\textcolor{bgctitle}{\fontfamily{cmss}\fontseries{bx}\fontshape{n}\fontsize{20.48}{0pt} \selectfont #1}}

\addtolength{\oddsidemargin}{-1.0cm}
\addtolength{\evensidemargin}{-1.0cm}
\addtolength{\headwidth}{2.0cm}
\addtolength{\textwidth}{2.0cm}

\setlength{\parindent}{0cm}

\renewcommand{\labelitemi}{$\circ$}
\renewcommand{\labelitemii}{$\diamond$}

\newcommand{\spaceline}[1][8pt]{\vskip #1}
\newcommand{\comment}[1]{\spaceline[5pt] \textcolor{fgcgray}{\scriptsize #1} \spaceline[15pt]}
\newcommand{\attrname}[1]{\textcolor{fgcgray}{\scriptsize #1}}


\makeatletter

\newcommand*{\project}[1]{\gdef\@project{#1}}
\newcommand*{\version}[1]{\gdef\@version{#1}}
\newcommand*{\home}[1]{\gdef\@home{#1}}
\newcommand*{\homeref}[1]{\gdef\@homeref{#1}}
\newcommand*{\prerequisite}[1]{\gdef\@prerequisite{#1}}
\newcommand*{\prerequisiteref}[1]{\gdef\@prerequisiteref{#1}}

\def\@maketitle{
  %\begin{titlepage}
  \begin{center}
    \colorbox{bgctitle}{
      \parbox{\textwidth}{
        \spaceline
        \centering{\titlefont{\@title}}
        \par
        \spaceline
      }
    }
    \colorbox{white}{
      \parbox{\textwidth}{
        \spaceline
        \centering{\inversetitlefont{\@project}}
        \par
        \spaceline
      }
    }
  \end{center}
  \spaceline[1.5em] {
    \begin{flushright}
    \begin{tabular}[t]{rl}
      \attrname{Projekt:} & \@project ~ \@version \\
      \attrname{Voraussetzung:} & \href{\@prerequisiteref}{\@prerequisite} \\
      \attrname{Autor:} & \@author \\
      \attrname{Home:} & \href{\@homeref}{\@home} \\
      \attrname{letzte "Anderung:} & \@date
    \end{tabular}
    \end{flushright}
    \par
  }
  \spaceline[5.5em]
  %\end{titlepage}
}

\setcounter{secnumdepth}{4}
\setcounter{tocdepth}{4}
	 
\newcounter{subsubsubsection}[subsubsection]
\def\subsubsubsectionmark#1{}
\def\thesubsubsubsection{\thesubsubsection .\arabic{subsubsubsection}}
\def\subsubsubsection{\@startsection{subsubsubsection}{4}{\z@}{-3.25ex plus -1 ex minus -.2ex}{1.5ex plus .2ex}{\normalsize\bf}}
\def\l@subsubsubsection{\@dottedtocline{4}{4.8em}{4.2em}}

\makeatother

\everytexdraw{
  \drawdim cm \linewd 0.01
  \arrowheadtype t:T
  \arrowheadsize l:0.2 w:0.2
  \setgray 0.5
}

\newcommand{\xheight}{0.6}
\newcommand{\xlength}{0.6}
\newcommand{\yheighta}{1.0}
\newcommand{\yheightb}{0.8}
\newcommand{\yheightc}{0.6}
\newcommand{\yheightd}{0.5}
\newcommand{\yheighte}{0.4}
\newcommand{\yheightf}{0.35}

\newcommand{\xhline}{\rlvec({\xlength} 0)}
\newcommand{\xharrow}{\ravec(0.7 0)}
\newcommand{\xnext}{%\rlvec(0.05 0) \lpatt(0.04 0.04) \rlvec(0.15 0) \lpatt()
}

\newcommand{\xtext}[3][\xheight]{
  \bsegment
    \bsegment
      \setsegscale 0.5
      \textref h:L v:C  \htext({\xheight} -0.1){#3}
    \esegment
    \setsegscale 0.5 \lvec(0 #1)
    \setsegscale 1
    \rlvec(#2 0) \rlvec(0 -#1) \rlvec(-#2 0) \lvec(0 0)
    \savepos(#2 0)(*@x *@y)
  \esegment
  \move(*@x *@y)
}

\newcommand{\bxtext}[3][\xheight]{
  \setgray{0.1}
  \linewd{0.026}
  \xtext[\xheight]{#2}{#3}
  \linewd{0.01}
  \setgray{0.5}
}

\newcommand{\xstartpage}{\bxtext{2.1}{Startseite}}
\newcommand{\xmainpage}{\bxtext{2.3}{Hauptseite}}
\newcommand{\xusermenu}{\bxtext{2.9}{Benutzermenu}}
\newcommand{\xgamelist}{\bxtext{2.2}{Spieleliste}}
\newcommand{\xportfolio}{\bxtext{2.0}{Portfolio}}
\newcommand{\xaccount}{\xtext{3.8}{Kennung per \textsl{eMail}}}


\begin{document}
  \lhead{\sc{Beispiel: Pflichtenheft Brettspiele}}
%\cfoot{-~\thepage~-}
\title{Beispiel: Pflichtenheft}
\project{Brettspiele}
\version{1.2}
\prerequisite{Lastenheft}
\prerequisiteref{http://www.stefan-baur.de/downloads/Lastenheft.pdf}
\author{Stefan K. Baur}
\home{www.stefan-baur.de}
\homeref{http://www.stefan-baur.de/cs.se.pflichtenheft.html}

\maketitle

\textcolor{warning}{Diese Datei zeigt NUR ein Beispiel eines Pflichtenheftes!}\\
\textcolor{warning}{Verwendung auf eigene Gefahr!}
%\thispagestyle{empty}
%~
%\newpage
%\tableofcontents
 \newpage
  \tableofcontents \newpage
  \section{Zielbestimmungen}

\comment{Welche Musskriterien, Wunschkriterien, Abgrenzungskriterien sind erforderlich?}

\textbf{Brettspiele} stellt einen Internetdienst dar, der das Spielen von Brettspielen zwischen Einzelpersonen erm�glicht.
Im Folgenden bezeichne \textit{Benutzer} und \textit{Spieler} sowohl die weiblichen, als auch die m�nnlichen Benutzer und Spieler.

\subsection{Musskriterien}

\begin{itemize}
  \item Der Benutzer-Account
    \begin{itemize}
      \item Der Internet-Benutzer kann sich selbst am System registrieren.
      \item Der Benutzer kann sich am System anmelden und vom System abmelden.
      \item Der Benutzer kann seine Kennung anfordern.
      \item Der Benutzer kann seine pers�nlichen Daten einsehen und �ndern, sowie deren Sichtbarkeit innerhalb der Spielgemeinschaft einstellen.
      \item Der Benutzer kann die pers�nlichen Daten anderer Benutzer einsehen, soweit diese sichtbar sind.
      \item Der Benutzer kann sein pers�nliches Profil einsehen, sowie die Sichtbarkeit innerhalb der Spielgemeinschaft einstellen.
      \item Der Benutzer kann das pers�nliche Profil anderer Benutzer einsehen, soweit dieses sichtbar ist.
      \item Der Benutzer kann seine Nutzungsoberfl�che nach eigenem Bedarf und Geschmack konfigurieren.
      \item Der Benutzer kann Konfigurationen der eigenen Nutzungsoberfl�che neu erstellen, speichern, l�schen, �ndern und wieder verwenden.
      \item Der Benutzer ist in Besitz einer eigenen Portfolio, in der er Konfigurationen, Benutzer, Spiele, Nachrichten und Notizen verwalten kann.
      \item Der Benutzer kann mit den Funktionen des Portfolios das System durchsuchen, die Suchergebnisse k�nnen dem Portfolio hinzugef�gt werden.
      \item Der Benutzer kann jeden einzelnen Eintrag im Portfolio kommentieren und bewerten.
      \item Die Benutzer k�nnen untereinander Nachrichten austauschen (Instant-Messaging).
    \end{itemize}
  \item Das Spiel
    \begin{itemize}
      \item Es stehen drei Spieltypen zur Verf�gung: \textit{M�hle}, \textit{Dame} und \textit{Schach}.
      \item Der Spieler kann Spiele beliebigen Typs er�ffnen.
      \item Der Spieler kann ein er�ffnetes Spiel aufnehmen.
      \item Die Spieler k�nnen sich untereinander zum Spiel herausfordern.
      \item Der Spieler muss gem�� den Spielregeln ziehen, ein Unentschieden anbieten/annehmen oder aufgeben bzw. gewinnen.
      \item Der Spieler kann bei jeder Spielaktion eine kurze Nachricht �bermitteln.
      \item Nach dem Beenden einer Partie wird das pers�nliche Profil gem�� der Spielst�rke des Spielers mit einer zum Spieltyp geh�renden Berechnungsfunktion aktualisiert.
    \end{itemize}
  \item Der Administrator
    \begin{itemize}
      \item Der Administrator konfiguriert die Betriebsparameter des Systems.
      %\item Der Administrator muss den Zugang aller Benutzer vor�bergehend sperren k�nnen,
      \item Der Administrator sichert die Datenbank.
    \end{itemize}
  \item Sonstiges
    \begin{itemize}
      \item Englisch als Verkehrssprache.
      \item Erweiterbarkeit des Systems weiterer europ�ischen Sprachen.
      \item Erweiterbarkeit des Systems weiterer Spieltypen \textit{(z.B. Go oder Backgammon)}.
      \item Erweiterbarkeit des Systems weiterer Bereiche \textit{(z.B. Spiele-Forum, Lehrb�cherverkauf)}.
      \item Erweiterbarkeit des Systems zur geb�hrenpflichten Nutzung f�r die Benutzer.
    \end{itemize}
\end{itemize}

\subsection{Wunschkriterien}

\begin{itemize}
  \item Die Benutzer k�nnen die Eintr�ge im Portfolio untereinander austauschen.
  \item Der Benutzer kann den Verlauf seiner beendeten Spiele einsehen.
  \item Zu jedem Spieltyp liegen die Spielregeln im System vor.
  \item Einfache Integration von Werbebannern, einstellbar durch Administrator.
\end{itemize}

\subsection{Abgrenzungskriterien}

\begin{itemize}
  \item Nur Fernspiele, also z.B. Fernschach und kein Blitzschach.
  \item Das System eignet sich nur f�r Zwei-Spieler-Spiele.
\end{itemize}
 \newpage
  \section{Produkteinsatz}

\comment{Welche Anwendungsbereiche (Zweck), Zielgruppen (Wer mit welchen Qualifikationen), Betriebsbedingungen (Betriebszeit, Aufsicht)?}

\subsection{Anwendungsbereiche}

Einzelpersonen verwenden diesen Dienst zum Spielen der oben angegebenen Brettspiele mit anderen Personen der Spielgemeinschaft.
Diese Plattform soll dem Einzelnen eine Kommunikation mit Gleichgesinnten erm�glichen,
um so ihre Fertigkeiten im Spiel verbessern zu k�nnen.

\subsection{Zielgruppen}

Personengruppen, die kurz zur Ablenkung z.B. in der Mittagspause, gerne an Fernspielen teilhaben,
in dem sie sich Gedanken �ber bevorstehende Spielz�ge machen k�nnen.\\
Diese Plattform ist f�r Einzelpersonen gedacht, die in ihrer knapp bemessenen Freizeit Schwierigkeiten haben,
ihrem Hobby z.B. Schach nachzugehen oder Gegner zu finden.\\
\\
Es werden Basiskenntnisse in Internetnutzung vorausgesetzt. Ebenso die Spielregeln des jeweiligen Spieltyps sollten vor der
Nutzung bekannt sein.\\
\\
Soweit keine weiteren Sprachen integriert sind, muss der Benutzer die Verkehrssprache \textit{Englisch} zumindest verstehen.

\subsection{Betriebsbedingungen}

Dieses System soll sich bez�glich der Betriebsbedingungen nicht wesentlich von anderen Internetdiensten bzw. -anwendungen unterscheiden.

\begin{itemize}
  \item Betriebsdauer: t�glich, 24 Stunden
  \item Wartungsfrei
  \item Die Sicherung der Datenbank muss manuell vom Administrator durchgef�hrt werden.
  \item Falls n�tig, ist der Administrator zur Schlichtung zwischen Benutzern verantwortlich.
\end{itemize}
 \newpage
  \section{Produktumgebung}

\comment{Welche Software, Hardware und Orgware wird ben�tigt?}

Das Produkt ist weitgehend unabh�ngig vom Betriebssystem, sofern folgende Produktumgebung vorhanden ist.

\subsection{Software}

\begin{itemize}
  \item Client
    \begin{itemize}
      \item \textbf{www-Browser} der neuesten Generation: Internet Explorer 6 und Mozilla 1.3 \textit{(keine textbasierten Browser)}
    \end{itemize}
  \item Server
    \begin{itemize}
      \item \textbf{PHP} \textit{(mind. Version 4.0.5)}
      \item \textbf{MySQL}-Datenbank
      \item \textbf{SMTP}-Server \textit{(eMail-f�hig)}
    \end{itemize}
\end{itemize}

\subsection{Hardware}

\begin{itemize}
  \item Client
    \begin{itemize}
      \item Internetf�higer Rechner
    \end{itemize}
  \item Server
    \begin{itemize}
      \item Internetf�higer Server
      \item Rechner, der die Anspr�che der o.g. Server-Software erf�llt
      \item Ausreichend Rechen- und Festplattenkapazit�t
    \end{itemize}
\end{itemize}

\subsection{Orgware}

\begin{itemize}
  \item Gew�hrleistung der permanenten Internetanbindung
  \item Administrator muss den Internetdienst starten und die Betriebsparameter konfigurieren
\end{itemize}
 \newpage
  \section{Funktionale Anforderungen}

\comment{Was leistet das Produkt aus Benutzersicht?}

\subsection{Benutzerfunktionen}

\subsubsection{Benutzer-Kennung}

Ein im System registrierter Benutzer kann das System erst nutzen, wenn er angemeldet ist.

\begin{description}
  \item[/F0010/]
    \textit{Registrieren:} Ein beliebiger Internet-Benutzer kann sich �ber die Start- bzw. Login-Seite des Systems
    schnell und bequem registrieren lassen. Zum Registrieren sind mindestens folgende Angaben erforderlich:
    \begin{itemize}
      \item gew�nschte \textbf{Kennung}
        \begin{itemize}
          \item gew�nschter \textbf{Benutzername}
          \item gew�nschtes \textbf{Passwort}
        \end{itemize}
      \item eigene bzw. private \textbf{eMail-Adresse}
    \end{itemize}
    Die Registrierung ist erfolgreich, wenn der \textit{Benutzername} und die \textit{eMail-Adresse}
    innerhalb des Systems jeweils eindeutig sind. Die \textit{eMail-Adresse} wird auf ihre G�ltigkeit gepr�ft.\\
    Mit dem erfolgreichen Abschie�en des Registrierungsvorgangs ist der neue Benutzer am System angemeldet,
    zudem erh�lt der Benutzer automatisch via \textit{eMail} vom System seine aktuelle Kennung.
  \item[/F0020/]
    \textit{Anmelden:} Ein bereits registrierter Benutzer kann sich �ber die Start- bzw. Login-Seite des Systems
    schnell und bequem anmelden \textit{(login)}. Dazu ist seine Kennung erforderlich:
    \begin{itemize}
      \item sein \textbf{Benutzername}
      \item sein \textbf{Passwort}
    \end{itemize}
    Alternativ zum \textit{Benutzernamen} kann der Benutzer seine \textit{eMail-Adresse} angeben.
  \item[/F0030/]
    \textit{Abmelden:} Der angemeldete Benutzer kann sich jeder Zeit wieder vom System \textbf{abmelden} \textit{(logout)}.
  \item[/F0040/]
    \textit{Kennung anfordern:} Falls ein bereits registrierter Benutzer seine Kennung oder sein \textbf{Passwort vergessen}
    haben sollte, so kann er seine korrekte Kennung �ber die Start- bzw. Login-Seite des Systems anfordern.
    Dem Benutzer wird unter Angabe
    \begin{itemize}
      \item seines \textbf{Benutzernamens} oder
      \item seiner \textbf{eMail-Adresse}
    \end{itemize}
    seine vollst�ndige Kennung automatisch via \textit{eMail} vom System zugesendet.
  \item[/F0050/]
    \textit{Passwort �ndern:} Der angemeldete Benutzer kann das Passwort seiner Kennung �ndern.
    Das neue Passwort muss zweimal angegeben werden, wobei sich diese Angaben nicht unterscheiden d�rfen.
    Nach erfolgreicher �nderung des Passwortes erh�lt der Benutzer automatisch via \textit{eMail} vom System seine aktuelle Kennung.
\end{description}

Der Benutzer kann seinen \textit{Benutzernamen} nicht �nderen.\\
\\
Im Folgenden sei der Benutzer stets am System angemeldet.

\subsubsection{Pers�nliche Daten}

Der Benutzer verf�gt �ber pers�nliche Daten \textit{(siehe /D010/)}, die er frei gestalten kann.

\begin{description}
  \item[/F0110/]
    \textit{Anzeige der eigenen, pers�nlichen Daten:}
    Der Benutzer kann sich seine pers�nlichen Daten vom System \textbf{vollst�ndig anzeigen} lassen.
  \item[/F0120/]
    \textit{�ndern der eigenen, pers�nlichen Daten:}
    Der Benutzer kann seine pers�nlichen Daten aktualisieren bzw. \textbf{�ndern}.
  \item[/F0130/]
    \textit{Sichtbarkeit der eigenen, pers�nlichen Daten:}
    Der Benutzer kann jeden einzelnen Eintrag seiner pers�nlichen Daten f�r die Spielgemeinschaft auf \textbf{sichtbar} bzw.
    \textbf{unsichtbar} setzen.
  \item[/F0140/]
    \textit{Anzeige der pers�nlichen Daten anderer Benutzer:}
    Der Benutzer kann sich von anderen Benutzern die pers�nlichen Daten anzeigen lassen,
    dabei k�nnen auf unsichtbar gesetzte Eintr�ge nicht gesehen werden.\\
    Im Gegensatz zu \textit{/F0110/} kann der Benutzer seine eigenen, pers�nlichen Daten auch auf diese Weise anzeigen lassen.
\end{description}


\subsubsection{Pers�nliche Konfiguration}

Die Nutzungsumgebung eines Benutzers ist das Layout, das Design, aber auch diverse logische Einstellungen,
die die individuelle Handhabung des Systems vereinfachen k�nnen.\\
Individuell einstellbar f�r den Benutzer sind:
\begin{itemize}
  \item die Farbgebung
  \item die Gliederung seiner Hauptseite \textit{(Anordnung von Men�, Spielfl�che und Spieleliste)}
\end{itemize}
Zudem kann der Benutzer noch einstellen, welche Informationen direkt nach dem \textit{Login} auf der Hauptseite angezeigt werden sollen.
Pers�nliche Konfigurationen k�nnen verwaltet werden \textit{(siehe Portfolio)}.

\begin{description}
  \item[/F0210/]
    \textit{Anzeige der pers�nlichen Konfiguration:}
    Der Benutzer kann sich alle einstellbaren Werte seiner pers�nlichen Konfiguration seiner Nutzungsumgebung vom System \textbf{anzeigen} lassen.
  \item[/F0220/]
    \textit{�ndern der pers�nlichen Konfiguration:}
    Der Benutzer kann alle einstellbaren Werte seiner pers�nlichen Konfiguration \textbf{�ndern}
    oder die voreingestellte Konfiguration wiederherstellen.
  \item[/F0230/]
    \textit{Speichern der pers�nlichen Konfiguration:}
    Der Benutzer kann seine pers�nliche Konfiguration \textbf{speichern} bzw. in seiner \textit{Portfolio} \textbf{sichern}.
  \item[/F0240/]
    \textit{L�schen der pers�nlichen Konfiguration:}
    Der Benutzer kann bereits gesicherte Konfigurationen aus seiner \textit{Portfolio} \textbf{entfernen}.
  \item[/F0250/]
    \textit{Wiederverwenden der pers�nlichen Konfiguration:}
    Der Benutzer kann bereits gesicherte Konfigurationen seiner \textit{Portfolio} \textbf{wiederverwenden}.
    Beim Wechseln der Konfiguration wird das gleichzeitige Sichern der aktuellen Konfiguration angeboten.
\end{description}

\subsubsection{Pers�nliches Profil}

Der Benutzer bzw. der Spieler verf�gt �ber ein pers�nliches Profil.
Dieses kann man in zwei Teile gliederen:
\begin{itemize}
  \item allgemeines Profil
    \begin{itemize}
      \item die H�ufigkeit des Erscheinens \textit{(Treue)}
    \end{itemize}
  \item Profil zu jedem Spieltyp \textit{(M�hle, Dame und Schach)}
    \begin{itemize}
      \item die Wertung \textit{(aktuelle Spielst�rke in Form einer Zahl)}
      \item die Anzahl der gespielten, gewonnenen, verlorenen und unentschiedenen Spiele
      \item die Anzahl der noch offenen Spiele
      \item die Auflistung von Auszeichnungen
        \begin{itemize}
          \item der beste
          \item der schlechteste
          \item der schnellste
          \item der langsamste
          \item der treueste
          \item der bekannteste Spieler \textit{(der Woche, des Monats und des Jahres)}
        \end{itemize}
    \end{itemize}
\end{itemize}

Das pers�nliche Profil wird je nach Bedarf vom System automatisch aktualisiert, die Wertung wird z.B. nach dem Beenden einer Partie aktualisiert.

\begin{description}
  \item[/F0310/]
    \textit{Anzeige des eigenen, pers�nlichen Profils:}
    Der Benutzer kann sich sein pers�nliches Profil f�r jeden Spieltyp \textbf{anzeigen} lassen.
  \item[/F0320/]
    \textit{Sichtbarkeit des eigenen, pers�nlichen Profils:}
    Der Benutzer kann jeden einzelnen Eintrag seines pers�nlichen Profils f�r die Spielgemeinschaft auf \textbf{sichtbar} bzw.
    \textbf{unsichtbar} setzen.
    Jedoch die Wertungen bleiben immer �ffentlich sichtbar.
  \item[/F0330/]
    \textit{Anzeige der pers�nlichen Profile anderer Benutzer:}
    Der Benutzer kann sich von anderen Benutzern die pers�nlichen Profile anzeigen lassen,
    dabei k�nnen auf unsichtbar gesetzte Eintr�ge nicht gesehen werden.\\
    Im Gegensatz zu \textit{/F0310/} kann der Benutzer sein eigenes, pers�nliches Profil auch auf diese Weise anzeigen lassen.
\end{description}


\subsection{Spielfunktionen}

Der angemeldete Benutzer ist in erster Linie ein Spieler.
Dieser Spieler hat stets eine Liste von eigenen, laufenden Spielen zur Verf�gung.\\
Ein Spieler kann nicht gegen sich selbst antreten.

\subsubsection{Initialisierung}

\begin{description}
  \item[/F0410/]
    \textit{Er�ffnung eines Spieles:}
    Der angemeldete Benutzer kann Spiele \textbf{er�ffnen}, ohne dabei einen anderen Spieler als Gegner angeben zu m�ssen.
    Ein er�ffnetes Spiel kann von anderen Spieleren unter dem Men�punkt \textbf{neue Spiele} angenommen werden \textit{/F0420/}.
  \item[/F0420/]
    \textit{Aufnahme eines Spieles:}
    Der angemeldete Benutzer kann bereits er�ffnete Spiele \textbf{aufnehmen} \textit{/F0410/}.
  \item[/F0430/]
    \textit{Herausfordern eines Gegners:}
    Der angemeldete Benutzer kann unter Angabe eines g�ltigen Benutzernamens einen anderen Benutzer zum Spiel \textbf{herausfordern}.
  \item[/F0440/]
    \textit{Annahme einer Herausforderung:}
    Der angemeldete Benutzer kann eine Herausforderung zum Spiel \textit{/F0430/} \textbf{annehmen}.
  \item[/F0450/]
    \textit{Ablehnen einer Herausforderung:}
    Der angemeldete Benutzer kann eine Herausforderung zum Spiel \textit{/F0430/} \textbf{ablehnen}.
\end{description}

\subsubsection{Spielverlauf}

\begin{description}
  \item[/F0510/]
    \textit{Zugm�glichkeit:}
    Der angemeldete Benutzer kann zu jedem Spiel einen Zug seiner Wahl gem�� der Spielregeln \textbf{ziehen}, vorausgesetzt er ist am Zug.
  \item[/F0520/]
    \textit{Gebot eines Unentschieden:}
    Der angemeldete Benutzer kann gem�� den Spielregeln ein Unentschieden \textit{(remis)} \textbf{anbieten}, vorausgesetzt er ist am Zug.
  \item[/F0530/]
    \textit{Annahme eines Unentschieden:}
    Der angemeldete Benutzer kann, wenn ihm ein Unentschieden angeboten wurde \textit{/F0520/}, das Unentschieden \textbf{annehmen}.
  \item[/F0540/]
    \textit{Ablehnen eines Unentschieden:}
    Der angemeldete Benutzer kann, wenn ihm ein Unentschieden angeboten wurde \textit{/F0520/}, das Unentschieden \textbf{ablehnen}.
  \item[/F0550/]
    \textit{Aufgabe eines Spieles:}
    Der angemeldete Benutzer kann ein laufendes Spiel \textbf{mit Aufgabe beenden}, vorausgesetzt er ist am Zug.
  \item[/F0560/]
    \textit{Zugbedingter Nachrichtenaustausch:}
    Der angemeldete Benutzer kann zu jedem Zug \textbf{eine Nachricht �bermitteln}.
\end{description}

\subsection{Portfolio-Funktionen}

Der Benutzer verf�gt �ber eine pers�nliche Sammel- und Dokumentenmappe,
die mit Such-, Sortier- und Dokumentationsfunktionen versehen ist.
Dieses Portfolio ist rein zur privaten Verwendung und deshalb nicht f�r die Spielgemeinschaft sichtbar.
Das Portfolio erm�glicht das individuelle Dokumentieren und Sammeln im System.\\
Der Benutzer kann grunds�tzlich alles Sammelbare in sein Portfolio aufnehmen.\\
Zum Sammelbaren z�hlen:
\begin{itemize}
  \item pers�nliche Konfigurationen
  \item andere Benutzer bzw. Spieler
  \item bereits beendete Spiele
  \item Nachrichten
  \item Notizen
\end{itemize}

\begin{description}
  \item[/F0610/]
    \textit{Anzeige des Portfolios:}
    Der Benutzer kann sich den Inhalt des pers�nlichen Portfolios \textbf{anzeigen} lassen.
  \item[/F0620/]
    \textit{Suche nach Benutzern:}
    Der Benutzer kann mit der Suchfunktion des Portfolios nach anderen Benutzern des Systems anhand einer beliebigen Kombination
    der folgenden Kriterien suchen:
    \begin{itemize}
      \item Benutzername
      \item Bereichsangabe der Wertungen zu gegebenen Spieltyp
    \end{itemize}
  \item[/F0630/]
    \textit{Suche nach Spielen:}
    Der Benutzer kann mit der Suchfunktion des Portfolios nach bereits gespielten Spielen des Systems anhand einer beliebigen Kombination
    der folgenden Kriterien suchen:
    \begin{itemize}
      \item Spieltyp \textit{(M�hle, Dame, Schach oder alle)}
      \item Anzahl der Spielz�ge
      \item Gewinnfarbe \textit{(Weiss, Schwarz oder unentschieden)}
    \end{itemize}
\end{description}

\subsection{Administratorfunktionen}

Der Administrator verf�gt �ber alle Benutzerfunktionen, und kann dar�berhinaus die Eigenschaften des Systems konfigurieren.
Zudem kann der Administrator Benutzer aus dem System verbannen,
sowie den Informationsaustausch \textit{(Instant-Messaging)} zwischen zwei Benutzern v�llig unterbinden,
sofern diese kein Spiel am Laufen haben.

\subsubsection{Systemverwaltung}

\begin{description}
  \item[/F1010/]
    \textit{Konfiguration:}
    Der angemeldete Administrator kann die Eigenschaften des Systems
    \begin{itemize}
      \item Sessiondauer eines Benutzers \textit{(Autologout)}
      \item Welche Werbebanner auf welchen Seiten angezeigt werden sollen
    \end{itemize}
    konfigurieren.
  \item[/F1020/]
    \textit{Statistiken:}
    Der angemeldete Administrator kann sich Statistiken
    \begin{itemize}
      \item Welcher Spieltyp wird am h�ufigsten verwendet
      \item Wieviele Benutzer registrieren sich pro Tag
      \item Wieviele Benutzer sind bzw. waren am System angemeldet
    \end{itemize}
    zur Benutzung des Systems anzeigen lassen.
\end{description}

\subsubsection{Benutzerverwaltung}

\begin{description}
  \item[/F1110/]
    \textit{Einsch�nkung der Benutzer:}
    Der angemeldete Administrator kann die Eigenschaften einzelner Benutzer unter Angabe einer zeitlichen Begrenzung einschr�nken.
    \begin{itemize}
      \item Er kann die M�glichkeit zum Nachrichtenaustausch \textit{(Instant-Messaging)} zweier Benutzer unterbinden.
      \item Er kann jeglichen Kontakt zwischen zwei Benutzer unterbinden, wobei bereits er�ffnete Spiele zu Ende gespielt werden m�ssen.
    \end{itemize}
    Eine v�llige Einschr�nkung eines Benutzers bedeutet die Verbannung eines Benutzers aus dem System.
    Dies ist besonders sinnvoll, wenn ein Benutzer bei einem kostenpflichtigen Account seinen Zahlungen nicht nachkommen sollte
    \textit{(Ausblick auf Folgeversion)}.
  \item[/F1120/]
    \textit{Einschr�nkungen restaurieren:}
    Der angemeldete Administrator kann die Eigenschaften einzelner Benutzer auch wieder manuell restaurieren.
\end{description}
 \newpage
  \section{Produktdaten}

\comment{Was speichert das Produkt (langfristig) aus Benutzersicht?}

Jeder Punkt \textbf{/D???/} stellt im Prinzip einen Datensatz dar.

\begin{description}
  \item[/D010/]
    \textit{Benutzerdaten:} Alle Informationen zu einem Benutzer:
    \begin{itemize}
      \item \textbf{BenutzerID} \textit{(eindeutig)}
      \item Kennung
        \begin{itemize}
          \item \textbf{Benutzername} \textit{(eindeutig)}
          \item \textbf{Passwort} \textit{(verschl�sselt)}
        \end{itemize}
      \item Pers�nliche Daten
        \begin{itemize}
          \item Informationen zur eigenen Person
            \begin{itemize}
              \item \textbf{Vorname}
              \item \textbf{Nachname}
              \item \textbf{Alter}
              \item \textbf{Geschlecht} \textit{(m�nnlich, weiblich)}
              \item kleines \textbf{Foto}
              \item \textbf{Begr��ungstext}
              \item \textbf{Slogan}
            \end{itemize}
          \item Kontaktinformationen
            \begin{itemize}
              \item \textbf{Stra�e und Hausnummer}
              \item \textbf{Postleitzahl}
              \item \textbf{Ort}
              \item \textbf{Land}
              \item \textbf{Telefon}
              \item \textbf{Fax}
              \item \textbf{eMail-Adresse} \textit{(eindeutig, g�ltig)}
              \item \textbf{Homepage}
            \end{itemize}
          \item \textbf{Sichtbarkeit} der einzelnen Eintr�ge der pers�nlichen Daten
        \end{itemize}
      \item Sonstige Daten
        \begin{itemize}
          \item \textbf{Registrierungsdatum} \textit{(Datum)}
          \item \textbf{letzte Anmeldung} \textit{(Datum)}
          \item \textbf{Besuchsz�hler}
          \item \textbf{Status} \textit{(Administrator, Benutzer)}
        \end{itemize}
    \end{itemize}
\end{description}

\begin{description}
  \item[/D020/]
    \textit{Profildaten:} Das pers�nliche Profil eines Benutzers zu einem Spieltyp.
    Die Anzahl der offenen, gewonnen, verlorenen und unentschiedenen Spiele
    wird zur Laufzeit ermittelt und nicht explizit im Profil gespeichert!
    \begin{itemize}
      \item \textbf{BenutzerID}
      \item \textbf{Spieletyp} \textit{(M�hle, Dame oder Schach)}
      \item \textbf{Wertungszahl}
      \item Sichtbarkeit
        \begin{itemize}
          \item \textbf{Offene Spiele}
          \item \textbf{Gewonnene Spiele}
          \item \textbf{Verlorene Spiele}
          \item \textbf{Unentschiedene Spiele}
        \end{itemize}
    \end{itemize}
\end{description}

\begin{description}
  \item[/D030/]
    \textit{Konfigurationsdaten:} Die pers�nliche Konfiguration eines Benutzers:
    \begin{itemize}
      \item \textbf{BenutzerID}
      \item Darstellungsfarben
        \begin{itemize}
          \item \textbf{Hintergrundfarbe}
          \item \textbf{Textfarbe}
          \item \textbf{Rahmenfarbe}
          \item \textbf{Spielfeldfarbe Weiss}
          \item \textbf{Spielfeldfarbe Schwarz}
        \end{itemize}
      \item Darstellungsschema
        \begin{itemize}
          \item \textbf{Menu} \textit{(links, oben oder rechts)}
          \item \textbf{Spielfeld} \textit{(zentrieren oder am Fensterrand)}
        \end{itemize}
      \item Ablauflogik
        \begin{itemize}
          \item vor Ausf�hrung eines Spielzuges Best�tigungsfenster anzeigen
          \item nach Ausf�hrung eines Spielzuges
            \begin{itemize}
              \item aktuelles Spiel erneut anzeigen oder
              \item alle eigenen offenen Spiele anzeigen oder
              \item das n�chste offene Spiel anzeigen
            \end{itemize}
        \end{itemize}
    \end{itemize}
\end{description}

\begin{description}
  \item[/D110/]
    \textit{Spieldaten:} Der aktuelle Status eines einzelnen Spieles:
    \begin{itemize}
      \item \textbf{SpielID} \textit{(eindeutig)}
      \item \textbf{Spieletyp} \textit{(M�hle, Dame oder Schach)}
      \item \textbf{Status} \textit{(Initiiert, Laufend, Weiss, Schwarz oder Unentschieden)}
      \item \textbf{Anzahl der Halbz�ge}
      \item Spieler 1 und 2 \textit{(Weiss und Schwarz)} jeweils:
        \begin{itemize}
          \item \textbf{BenutzerID}
          \item \textbf{Summe der Bedenkzeit} \textit{(X Minuten)}
        \end{itemize}
      \item Sonstige Daten
        \begin{itemize}
          \item \textbf{Halbzugbedenkzeit} \textit{(X Tage)}
          \item \textbf{Toleranzzeit} \textit{(X Tage)}
          \item \textbf{Initialisierungsdatum} \textit{(Datum)}
          \item \textbf{letzter Halbzug} \textit{(Datum)}
        \end{itemize}
    \end{itemize}
\end{description}

\begin{description}
  \item[/D111/]
    \textit{Zugdaten:} Einzelner Zug eines Spiels:
    \begin{itemize}
      \item \textbf{SpielID}
      \item \textbf{Zugnummer}
      \item \textbf{Anfangskoordinate}
      \item \textbf{Zielkoordinate}
    \end{itemize}
\end{description}
 \newpage
  \section{Nichtfunktionale Anforderungen}

\comment{Welche zeit- und umfangsbezogenen Anforderungen gibt es?}

\begin{description}
  \item[/NF100/]
    \textit{Schiedsrichter:}
    Jeder Spieltyp des Systems verf�gt �ber einen eigenen Schiedsrichter, der automatisch erkennt, ob ein Zug g�ltig ist.
    Zum anderen muss dieser Schiedsrichter erkennen, ob ein Spielende mit dem aktuellen Zug erreicht wurde.
    Der Schiedsrichter ist kein Mensch, sondern ein Objekt, welches �ber die Regeln des jeweiligen Spieltyps verf�gt.
  \item[/NF200/]
    \textit{Akkumulation:}
    Bei fehlererzeugenden Eingaben erh�lt der Benutzer als Fehlermeldung eine Auflistung aller eingegebenen Fehler.
  \item[/NF210/]
    \textit{Toleranz:}
    Bei fehlererzeugenden Eingaben muss der Benutzer die M�glichkeit haben, eine Korrektur der Eingabedaten vorzunehmen,
    ohne Eingaben wiederholt eingeben zu m�ssen.
\end{description}

 \newpage
  \section{Globale Testf�lle}

\comment{Was sind typische Szenarien, die das Produkt erf�llen muss?}

Jede Produktfunktion \textit{/F????/} wird anhand von konkreten Testf�llen \textit{/T????/} getestet.\\
Die dabei verwendeten Namen werden rein zuf�llig gew�hlt.

\begin{description}
  \item[/T0010/]
    \textit{Registrieren:}
    Herr Tim Testmann registriert sich mit dem gew�nschten Benutzernamen \textit{testmann}
    und dem Passwort \textit{testtest} und der EMailadresse \textit{tim@testmann.de} am System.\\
    Frau Beate Betamuster registriert sich ebenfall mit dem gew�nschten Benutzernamen \textit{betamuster}
    und dem Passwort \textit{betabeta} und der EMailadresse \textit{info@stefan-baur.de} am System.
  \item[/T0020/]
    \textit{Anmelden:}
    Tim Testmann meldet sich am System unter Benutzung seines Benutzernamens und Passwortes am System an.
  \item[/T0030/]
    \textit{Abmelden:}
    Tim Testmann meldet sich vom System wieder ab.
  \item[/T0040/]
    \textit{Kennung anfordern:}
    Beate Betamuster hat ihre Kennung vergessen und fordert unter Angabe ihrer EMailadresse ihre Kennung an.
  \item[/T0050/]
    \textit{Passwort �ndern:}
    Beate Betamuster �ndert sodann ihr Passwort \textit{betabeta} in \textit{testbeta} ab.
  \item[/T????/]
    ...
\end{description}

\textcolor{warning}{Es muss zu jeder weiteren Produktfunktion ein konkreter Testfall hinzugef�gt werden ...}
 \newpage
  \section{Systemmodelle} \newpage
  \subsection{Szenarien} \newpage
  \subsection{Anwendungsf�lle} \newpage
  \subsection{Objektmodell} \newpage
  \subsection{Dynamische Modelle} \newpage
  \subsection{Benutzerschnittstelle} \newpage
  \section{Glossar}

\comment{Definition aller wichtigen Begriffe zur Sicherstellung einer einheitlichen Terminologie.}

\begin{description}
  \item[Fernspiele]
    sind Spiele, die eine Bedenkzeit von mindestens einen Tag haben. Beim Fernschach oder Briefschach haben die Spieler f�r jeden
    Zug viele Tage Bedenkzeit.
  \item[Hauptseite]
    ist die Seite, auf die der Benutzer kommt, wenn er sich erfolgreich am System angemeldet hat.
  \item[Kennung]
    ist das Tupel, das ein Benutzer zur Anmeldung an das System ben�tigt: \textit{Benutzername} und \textit{Passwort}.
  \item[Konfiguration]
    Jeder einzelne Benutzer kann seine Nutzungsoberfl�che individuell gestalten.
    Diese Einstellung wird als \textit{pers�nliche Konfiguration} bezeichnet.\\
    Der Administrator konfiguriert das System f�r alle Benutzer \textit{(Konfiguration des Administrators)}.
  \item[�ffentlich]
    Unter \textit{�ffentlich} versteht man die Lesbarkeit nur innerhalb der Spielgemeinschaft des Systems,
    soweit nicht n�her beschrieben.
  \item[Portfolio]
    ist die Sammelmappe bzw. die Dokumentenmappe eines Benutzers.
    Darin kann der Benutzer all seine f�r wichtig empfundenen Informationen aufnehmen,
    wie z.B. interessante Spiele oder starke Gegner, sowie Notizen und Nachrichten.
    Das Portfolio kann als pers�nliche Datenbank betrachtet werden.
  \item[registrieren]
    Erstmaliges Anmelden eines beliebigen Internet-Benutzers, f�r den noch keine Kennung f�r das System vorliegt.
  \item[Spielgemeinschaft]
    Die Menge aller spielenden Benutzer des Systems.
  \item[Startseite]
    bzw. Loginseite ist die Seite die angezeigt wird, wenn ein beliebiger Internet-Benutzer auf den Internetdienst gelangt.
  \item[System]
    ist ein Synonym f�r den Internetdienst \textit{Brettspiele}, soweit nicht n�her beschrieben.
  \item[Verkehrssprache]
    ist die Sprache des Systems wie z.B. zur Kommunikation zwischen Benutzern.
\end{description}

\end{document}
