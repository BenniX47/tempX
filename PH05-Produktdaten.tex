\section{Produktdaten}

\comment{Was speichert das Produkt (langfristig) aus Benutzersicht?}

Jeder Punkt \textbf{/D???/} stellt im Prinzip einen Datensatz dar.

\begin{description}
  \item[/D010/]
    \textit{Benutzerdaten:} Alle Informationen zu einem Benutzer:
    \begin{itemize}
      \item \textbf{BenutzerID} \textit{(eindeutig)}
      \item Kennung
        \begin{itemize}
          \item \textbf{Benutzername} \textit{(eindeutig)}
          \item \textbf{Passwort} \textit{(verschl�sselt)}
        \end{itemize}
      \item Pers�nliche Daten
        \begin{itemize}
          \item Informationen zur eigenen Person
            \begin{itemize}
              \item \textbf{Vorname}
              \item \textbf{Nachname}
              \item \textbf{Alter}
              \item \textbf{Geschlecht} \textit{(m�nnlich, weiblich)}
              \item kleines \textbf{Foto}
              \item \textbf{Begr��ungstext}
              \item \textbf{Slogan}
            \end{itemize}
          \item Kontaktinformationen
            \begin{itemize}
              \item \textbf{Stra�e und Hausnummer}
              \item \textbf{Postleitzahl}
              \item \textbf{Ort}
              \item \textbf{Land}
              \item \textbf{Telefon}
              \item \textbf{Fax}
              \item \textbf{eMail-Adresse} \textit{(eindeutig, g�ltig)}
              \item \textbf{Homepage}
            \end{itemize}
          \item \textbf{Sichtbarkeit} der einzelnen Eintr�ge der pers�nlichen Daten
        \end{itemize}
      \item Sonstige Daten
        \begin{itemize}
          \item \textbf{Registrierungsdatum} \textit{(Datum)}
          \item \textbf{letzte Anmeldung} \textit{(Datum)}
          \item \textbf{Besuchsz�hler}
          \item \textbf{Status} \textit{(Administrator, Benutzer)}
        \end{itemize}
    \end{itemize}
\end{description}

\begin{description}
  \item[/D020/]
    \textit{Profildaten:} Das pers�nliche Profil eines Benutzers zu einem Spieltyp.
    Die Anzahl der offenen, gewonnen, verlorenen und unentschiedenen Spiele
    wird zur Laufzeit ermittelt und nicht explizit im Profil gespeichert!
    \begin{itemize}
      \item \textbf{BenutzerID}
      \item \textbf{Spieletyp} \textit{(M�hle, Dame oder Schach)}
      \item \textbf{Wertungszahl}
      \item Sichtbarkeit
        \begin{itemize}
          \item \textbf{Offene Spiele}
          \item \textbf{Gewonnene Spiele}
          \item \textbf{Verlorene Spiele}
          \item \textbf{Unentschiedene Spiele}
        \end{itemize}
    \end{itemize}
\end{description}

\begin{description}
  \item[/D030/]
    \textit{Konfigurationsdaten:} Die pers�nliche Konfiguration eines Benutzers:
    \begin{itemize}
      \item \textbf{BenutzerID}
      \item Darstellungsfarben
        \begin{itemize}
          \item \textbf{Hintergrundfarbe}
          \item \textbf{Textfarbe}
          \item \textbf{Rahmenfarbe}
          \item \textbf{Spielfeldfarbe Weiss}
          \item \textbf{Spielfeldfarbe Schwarz}
        \end{itemize}
      \item Darstellungsschema
        \begin{itemize}
          \item \textbf{Menu} \textit{(links, oben oder rechts)}
          \item \textbf{Spielfeld} \textit{(zentrieren oder am Fensterrand)}
        \end{itemize}
      \item Ablauflogik
        \begin{itemize}
          \item vor Ausf�hrung eines Spielzuges Best�tigungsfenster anzeigen
          \item nach Ausf�hrung eines Spielzuges
            \begin{itemize}
              \item aktuelles Spiel erneut anzeigen oder
              \item alle eigenen offenen Spiele anzeigen oder
              \item das n�chste offene Spiel anzeigen
            \end{itemize}
        \end{itemize}
    \end{itemize}
\end{description}

\begin{description}
  \item[/D110/]
    \textit{Spieldaten:} Der aktuelle Status eines einzelnen Spieles:
    \begin{itemize}
      \item \textbf{SpielID} \textit{(eindeutig)}
      \item \textbf{Spieletyp} \textit{(M�hle, Dame oder Schach)}
      \item \textbf{Status} \textit{(Initiiert, Laufend, Weiss, Schwarz oder Unentschieden)}
      \item \textbf{Anzahl der Halbz�ge}
      \item Spieler 1 und 2 \textit{(Weiss und Schwarz)} jeweils:
        \begin{itemize}
          \item \textbf{BenutzerID}
          \item \textbf{Summe der Bedenkzeit} \textit{(X Minuten)}
        \end{itemize}
      \item Sonstige Daten
        \begin{itemize}
          \item \textbf{Halbzugbedenkzeit} \textit{(X Tage)}
          \item \textbf{Toleranzzeit} \textit{(X Tage)}
          \item \textbf{Initialisierungsdatum} \textit{(Datum)}
          \item \textbf{letzter Halbzug} \textit{(Datum)}
        \end{itemize}
    \end{itemize}
\end{description}

\begin{description}
  \item[/D111/]
    \textit{Zugdaten:} Einzelner Zug eines Spiels:
    \begin{itemize}
      \item \textbf{SpielID}
      \item \textbf{Zugnummer}
      \item \textbf{Anfangskoordinate}
      \item \textbf{Zielkoordinate}
    \end{itemize}
\end{description}
